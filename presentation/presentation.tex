\documentclass{beamer}

\usepackage[german]{babel}

\usepackage{tikz}
\usetikzlibrary{trees}

\usetikzlibrary{calc}
\usetikzlibrary{bending}
\usetikzlibrary{decorations.pathreplacing}

\pgfdeclarelayer{bg}
\pgfsetlayers{bg,main}

\tikzset{
	db/.pic={
		\draw[white, fill=white](-.6,0) rectangle (.6,1.4);
		\draw[white, fill=white](0,1.4) ellipse [x radius=.6,y radius=.15];
		\foreach \y in {0,.5,1}
		{
			\draw[thick, fill=white]
				(-0.6,\y) to [looseness=0.5,bend right=90] ++(1.2,0)
						  to ++(0,0.4)
						  to [looseness=0.5,bend left=90] ++(-1.2,0)
						  to ++(0,-0.4);
			\draw[thick] (-0.6,\y+0.4) edge[looseness=0.5,bend left=90] ++(1.2,0);
		}
	}
}


\title{Git \& \textrm{\LaTeX}-Einführung}
\author{
    Noah Kälin\inst{1},
    Simon Walker\inst{1},
	Naoki Pross\inst{1}
}
\institute[HSR]{\inst{1}Hochschule f\"ur Technick Rapperswil}

\date{\today}

\usetheme{Malmoe}
\usecolortheme{seagull}

\AtBeginSection[]
{
	\begin{frame}[shrink]{Inhaltsverzeichnis}
		\tableofcontents[
			currentsection,
			hideothersubsections,
			sectionstyle=show/shaded,
		]
	\end{frame}
}

\begin{document}

\frame{\titlepage}

\part{Git}

\section{Warum Git}

\subsection{Warum}
\begin{frame}{Was m\"ochten wir?}
	\begin{block}{Das Problem}
		Dateien zwischen Machine synchronisieren und sie mit mehr Leute
		bearbeiten. Spezifischer Textdateien bearbeiten (z.B. Quellcode)
	\end{block}
	\pause

	\begin{block}{Eine L\"osung}
	\centering
	\vspace{1em}
	\includegraphics[width=0.25\textwidth]{res/git}\\
	\vspace{1em}

	(Engl. f\"ur Bl\"odmann) Entwickelt von Linus Torvalds
	\end{block}
\end{frame}

\subsection{Vorteile}
\begin{frame}{Was ist git?}
	\begin{block}{Was wir brauchen}
	\begin{itemize}
		\item Synchronisation
		\item Team Datei Bearbeitung
		\item Problemlose Offline-Nutzung
	\end{itemize}
	\end{block}
	\pause

	\begin{block}{Was kriegen wir extra}
	\begin{itemize}
		\item Geschichte jedes Dokuments im Projekt
		\item Kryptographische Sicherheit der Projektgeschichte
		\item Gemeinsamer Dateizugriff ohne zentraler Server 
	\end{itemize}
	\end{block}
\end{frame}

\begin{frame}[c]{Bemerkung}
	\centering
	{\LARGE  
	Git l\"ost ein komplex Problem \\
	daher ist Git auch komplex
	}

	{\footnotesize
		But don't worry it's not \emph{too} hard
	}
\end{frame}

\section{Git lernen}

\subsection{Repository}
\begin{frame}{Begriff: Repository}
\begin{minipage}{.55\linewidth}
\resizebox{\textwidth}{!}{\begin{tikzpicture}[
	grow via three points={
		one child at (0.8,-0.7) and two children at (0.8,-0.7) and (0.8,-1.4)
	},
	edge from parent path={
	($(\tikzparentnode\tikzparentanchor)+(.4cm,0pt)$) |- (\tikzchildnode\tikzchildanchor)
	},
	growth parent anchor=west,
	parent anchor=south west,
  ]
	{\tikzset{
    every node/.style={
        draw=lightgray,
		 fill=lightgray!40,
        thick,
        anchor=west,
        inner sep=3pt,
        minimum size=1pt,
		 font=\ttfamily,
	}}
	\node (P) at (-6, 4) {Project/}
	child { node[draw=red, fill=red!10] {.git/} }
	child { node {src/}
		child { node [draw=none, fill=none] {main.c} }
		child { node [draw=none, fill=none] {build.ninja} }
		child { node {\ldots} }
    }
    child [missing] {}
    child [missing] {}
    child [missing] {}
    child { node {release/}
        child { node [draw=none, fill=none] {magic} }
        child { node [draw=none, fill=none] {awesome.a} }
    };}

    \visible<2->{
	\draw[black] pic{db} node (DB) {};
	\draw[ultra thick, darkgray, ->]
		($(P.east)+(.2,0)$) to[out=0, in=90] ($(DB.north)+(0,1.6)$)
	;
	\node[below=.3cm, align=center] (DB.south) {Repository};
	}
\end{tikzpicture}}
\end{minipage}\hfill%
\begin{minipage}{.4\linewidth}
\pause
F\"ur Git ein \emph{Repo} ist eine Verzeichnis mit einer spezieller (unsichtbar)
Unterverzeichnis \texttt{\textcolor{red}{.git}}

\vspace{1em}

Man soll \textbf{nie} \texttt{.git} l\"oschen.
\end{minipage}
\end{frame}

\subsection{Commit}
\begin{frame}{Snapshots}
    \begin{center}
    \begin{tikzpicture}
    \node at (-1, 2) [draw, fill=gray!15, thick, font=\ttfamily, left] (P) {Project/};
    \pause
    \draw[black, thick, ->] (P.north east) ++(.2,-.2) arc (-90:180:.5)
        node[pos=.6, above]{bearbeiten};
    \draw[black, thick, ->] (0, 0) -- (7.2, 0) node [below] {Zeit};
    \foreach \x/\t in {0/16:04:28, 2/16:37:01, 4/17:15:44, 6/18:01:03}
    {
        \pause
        \draw[color=black, fill=magenta, thick] (\x, 0) circle (.12)
            node[below=5pt, anchor=east, rotate=70, font=\ttfamily, align=right] {\tiny 2019-09-18\\\small\t};
        \draw[thick, ->, gray] (-.8, 2) to[bend left, in=120] (\x, .2);
    }
    \end{tikzpicture}
    \end{center}
\end{frame}

\begin{frame}{Begriff: Commit}
    \begin{block}{Ein Commit enthalt}
    \begin{itemize}
        \item Die \"Anderungen der Dateien \textbf{``snapshot''} \pause
        \item Autor Name + Email \pause
        \item Zeitstempel \pause
        \item \textbf{Eine Beschreibung der \"Anderungen} \pause
        \item kryptographisches Hash der Dateien
    \end{itemize}
    \end{block}

    \pause
	\begin{center}
	\textbf{
		Commits \textcolor{red}{k\"onnen nicht} \"andert werden, \\
    	weil sie ``Geschichte'' des Projekts sind.
	}
    \end{center}
\end{frame}

\begin{frame}[fragile]{Commit Beispiel}
\centering\footnotesize
\begin{verbatim}
commit e584e04c5f8ffb14e50c701c1fd8178457a51743
Author: Nao Pross <naopross@thearcway.org>
Date:   Tue Jan 22 04:31:30 2019 +0100

    Add test for task and job, fix bug in job

    By being a std::set job did not allow to add duplicate
    elements, changing it to a std::multiset fixes the issue.
\end{verbatim}
\end{frame}

\begin{frame}[c]{Aber was ist ein Commit?}
	\begin{center}
	{\LARGE
		Commit = ``Logical Unit of Work''
	}

	\vspace{2em}
	Ein Commit kann auch sehr klein sein. \\
	Generell kann man sagen: 
	\vspace{2em}

	{\LARGE
		je mehr Commits, desto besser
	}
	\end{center}
\end{frame}

\subsection{Remote}
\begin{frame}{Begriff: Remote}\begin{center}
    \begin{minipage}{.65\linewidth}
	\begin{tikzpicture}
		\draw[thick]
			(0,0) pic{db} node(DB1){}
			(DB1) ++(4.5, 0) pic{db} node (DB2){};

		\begin{pgfonlayer}{bg}
		\draw[blue!80, fill=blue!10]
			(DB1.north west) ++(-1, 3) node[
				black, anchor=north west, below=.75cm, right=.15cm, align=left
			]{Laptop\\{\footnotesize (Sie)}} rectangle ($(DB1.south east) + (1, -1)$);
		\draw[red!80, fill=red!10]
			(DB2.north west) ++(-1, 3) node[
				black, anchor=north west, below=.75cm, right=.1cm, align=left
			]{Server\\{\footnotesize (z.B. Github)}} rectangle ($(DB2.south east) + (1, -1)$);

		\end{pgfonlayer}
		\visible<3->{
		\draw[]
			(DB2.north west) ++(-1, 3) node[
				black, anchor=south west, above=.25cm, right=.1cm, align=left
			]{\texttt{origin}};
		}

		\visible<4->{
		\draw[thick] (DB1) ++(1,  .2) edge[<-, bend right] 
			node[pos=.5, below]{\texttt{pull}} ++(2.5, 0);
		\draw[thick] (DB1) ++(1, 1.2) edge[->, bend left] 
			node[pos=.5, above]{\texttt{push}}  ++(2.5, 0);
		}
	\end{tikzpicture}
    \end{minipage}
	\pause
	\begin{minipage}[c]{.3\linewidth}
	Ein \emph{Remote} ist ein Clone (Kopie) des Repos auf eine andere Maschine
	\\
	\pause

	Remote k\"onnen ein name haben, z.B. \texttt{origin}
	\end{minipage}
\end{center}
\end{frame}

\begin{frame}{Zu einem Remote synchronisieren}
\begin{center}
\resizebox{.9\textwidth}{!}{
\begin{tikzpicture}
	\node (rem) at (0, 0) {}
					(.5,0) node[above=.2cm] {\texttt{origin}};

	\node (loc) at (0,-2.5) {}
					(.5,-2.5) node[above=.2cm] {Local};

	\begin{pgfonlayer}{bg}
	\draw[red, fill=red!10] ($(rem)+(-.5,1)$) rectangle ($(rem)+(6.5,-1)$);
	\draw[blue, fill=blue!10] ($(loc)+(-.5,1)$) rectangle ($(loc)+(6.5,-1)$);
	\end{pgfonlayer}

	\foreach \x in {0,...,2}
	{
		\draw[thick, black, ->] ($(rem)+(\x,0)$) edge ++(1, 0);
		\draw[thick, black, fill=orange] ($(rem)+(\x, 0)$) circle [radius=.1];

		\draw[thick, black, ->] ($(loc)+(\x, 0)$) edge ++(1, 0);
		\draw[thick, black, fill=orange] ($(loc)+(\x, 0)$) circle [radius=.1];
	}

	\foreach \x in {3,...,4}
	{
		\draw[thick, black, ->] ($(loc)+(\x, 0)$) edge ++(1, 0);
		\draw[thick, black, fill=magenta] ($(loc)+(\x, 0)$) circle [radius=.1];
	}

	\draw[thick, decorate, decoration={brace, mirror}] (2.5, -2.75) 
		-- node[below=.15cm]{\footnotesize Neue Commits} (4.5,-2.75);
	\pause

	\draw[ultra thick, ->, darkgray] (3, -2.25) to[bend left]
		node[pos=.5, left]{\texttt{push}} (3,-.2);
	
	\foreach \x in {3,...,4}
	{
		\draw[thick, black, ->] ($(rem)+(\x,0)$) edge ++(1, 0);
		\draw[thick, black, fill=magenta] ($(rem)+(\x, 0)$) circle [radius=.1];
	}
\end{tikzpicture}
}
\end{center}
\end{frame}

\begin{frame}{Von einem Remote synchronisieren}
\centering
\resizebox{.9\textwidth}{!}{
\begin{tikzpicture}
	\node (rem) at (0, 0) {}
					(.5,0) node[above=.2cm] {\texttt{origin}};

	\node (loc) at (0,-2.5) {}
					(.5,-2.5) node[above=.2cm] {Local};

	\begin{pgfonlayer}{bg}
	\draw[red, fill=red!10] ($(rem)+(-.5,1)$) rectangle ($(rem)+(6.5,-1)$);
	\draw[blue, fill=blue!10] ($(loc)+(-.5,1)$) rectangle ($(loc)+(6.5,-1)$);
	\end{pgfonlayer}

	\foreach \x in {0,...,2}
	{
		\draw[thick, black, ->] ($(rem)+(\x,0)$) edge ++(1, 0);
		\draw[thick, black, fill=orange] ($(rem)+(\x, 0)$) circle [radius=.1];

		\draw[thick, black, ->] ($(loc)+(\x, 0)$) edge ++(1, 0);
		\draw[thick, black, fill=orange] ($(loc)+(\x, 0)$) circle [radius=.1];
	}

	\foreach \x in {3,...,4}
	{
		\draw[thick, black, ->] ($(rem)+(\x,0)$) edge ++(1, 0);
		\draw[thick, black, fill=green!60!black] ($(rem)+(\x, 0)$) circle [radius=.1];
	}

	\draw[thick, decorate, decoration={brace}] (2.5, .25) 
		-- node[above=.15cm]{\footnotesize Neue Commits} (4.5,.25);
	\pause

	\draw[ultra thick, <-, darkgray] (3, -2.25) to[bend left]
		node[pos=.5, left]{\texttt{pull}} (3,-.2);

	\foreach \x in {3,...,4}
	{
		\draw[thick, black, ->] ($(loc)+(\x,0)$) edge ++(1, 0);
		\draw[thick, black, fill=green!60!black] ($(loc)+(\x, 0)$) circle [radius=.1];
	}
\end{tikzpicture}
}
\end{frame}

\subsection{Merge}
\begin{frame}{Merge}
\begin{center}
\resizebox{.9\textwidth}{!}{
\begin{tikzpicture}
	\node (rem) at (0, 0) {}
					(.5,0) node[above=.2cm] {\texttt{origin}};

	\node (loc) at (0,-2.5) {}
					(.5,-2.5) node[above=.2cm] {Local};

	\begin{pgfonlayer}{bg}
	\draw[red, fill=red!10] ($(rem)+(-.5,1)$) rectangle ($(rem)+(6.5,-1)$);
	\draw[blue, fill=blue!10] ($(loc)+(-.5,1)$) rectangle ($(loc)+(6.5,-1)$);
	\end{pgfonlayer}

	\foreach \x in {0,...,2}
	{
		\draw[thick, black, ->] ($(rem)+(\x,0)$) edge ++(1, 0);
		\draw[thick, black, fill=orange] ($(rem)+(\x, 0)$) circle [radius=.1];

		\draw[thick, black, ->] ($(loc)+(\x, 0)$) edge ++(1, 0);
		\draw[thick, black, fill=orange] ($(loc)+(\x, 0)$) circle [radius=.1];
	}

	\foreach \x in {3,...,4}
	{
		\draw[thick, black, ->] ($(rem)+(\x,0)$) edge ++(1, 0);
		\draw[thick, black, fill=green!60!black] ($(rem)+(\x, 0)$) circle [radius=.1];

		\draw[thick, black, ->] ($(loc)+(\x, 0)$) edge ++(1, 0);
		\draw[thick, black, fill=magenta] ($(loc)+(\x, 0)$) circle [radius=.1];
	}

	\pause
	\draw[thick, decorate, decoration={brace}] (2.5, .25) 
		-- node[above=.15cm]{\footnotesize Neue Commits} (4.5,.25);
	\draw[thick, decorate, decoration={brace, mirror}] (2.5, -2.75) 
		-- node[below=.15cm]{\footnotesize Neue Commits} (4.5,-2.75);

	\pause
	\draw[ultra thick, <-, darkgray] (3, -2.25) to[bend left]
		node[pos=.5, left]{\texttt{fetch}} (3,-.2);
\end{tikzpicture}
}
\end{center}
\end{frame}

\begin{frame}{Begriff: 3-Way Merge}
\begin{center}
\resizebox{.9\textwidth}{!}{
\begin{tikzpicture}
	\node (loc) at (0, 0) {}
					(.5,0) node[above=.2cm] {Local};

	\begin{pgfonlayer}{bg}
	\shade[top color=red!10, bottom color=blue!20] ($(loc)+(-.5,1)$) rectangle ($(loc)+(6.5,-2)$);
	\draw[blue] ($(loc)+(-.5,1)$) rectangle ($(loc)+(6.5,-2)$);
	\end{pgfonlayer}

	\foreach \x in {0,...,2,}
	{
		\draw[thick, black, fill=orange] ($(loc)+(\x, -1)$) circle [radius=.1] node[] (C\x) {};
	}

	\foreach \x in {3,...,4}
	{
		\draw[thick, black, fill=magenta] ($(loc)+(\x, -1)$) circle [radius=.1] node (CL\x) {};
		\draw[thick, black, fill=green!60!black] ($(loc)+(\x, 0)$) circle [radius=.1] node (CR\x) {};
	}

	\draw[thick, black]
		(C0) node[above=.4cm, right=.1] {\tiny\texttt{master}} to (C1) to (C2)
		(C2) to[in=200, out=30] (CR3) node [above=.5cm, right=.1] {\tiny\texttt{origin/master}}
		(CR3) to (CR4)
		(C2) to (CL3)
		(CL3) to (CL4)
	;
	
	\pause

	\draw[thick, fill=orange, black] ($(loc)+(5,-1)$) circle [radius=.1] node (C5) {};

	\draw[thick, black]
		(CR4) to[out=-20, in=160] (C5)
		(CL4) to (C5) to (6,-1)
	;

	\pause

	\draw[ultra thick, darkgray, ->]
		(4, -2.5) node[anchor=east] (T0) {\footnotesize ``True'' Merge} to[in=-90, out=0] (C5)
	;

	\node[] (T1) at ($(T0)+(0,-.4)$) {
		\tiny\texttt{Merge branch 'origin/master' into master}
	};

\end{tikzpicture}
}
\end{center}
\end{frame}

\begin{frame}[fragile]{Konflikte}
	\begin{block}{}
	D.h. Was passiert wenn 2 Leute die gleiche Linie in einem Dokument bearbeiten?
	\\
	\pause
	\vspace{1em}
	Git kann nicht wissen welche version besser ist, und so fragt dir.
	\end{block}
	\begin{verbatim}
...
Here is a line that nobody touched
<<<<<<< HEAD
I have edited this line
=======
Someone else has also edited this line
>>>>>>> origin/master
...\end{verbatim}
\end{frame}
\begin{frame}{Warum ``3-Way''?}
	\begin{center}
	\begin{figure}
	\fbox{\includegraphics[width=1\textwidth]{res/smerge}}
	\caption{Sublime Merge}
	\end{figure}
	\end{center}

	\pause
	\begin{center}
	\Large
	\texttt{pull} ist ein Alias f\"ur \texttt{fetch} + \texttt{merge}
	\end{center}
\end{frame}

\subsection{Branch}
\begin{frame}[c]{Branch}
	\begin{center}
	\scalebox{-1}[-1]{\includegraphics[width=\textwidth]{res/treebranch}}
	\end{center}
\end{frame}

\begin{frame}{Begriff: Branch}
\begin{center}
\resizebox{\textwidth}{!}{
\begin{tikzpicture}
	\node (loc) at (0, 0) {}
					(.5,0) node[above=.2cm] {Local};

	\begin{pgfonlayer}{bg}
	\draw[blue, fill=blue!10] ($(loc)+(-.5,1)$) rectangle ($(loc)+(6.5,-2)$);
	\end{pgfonlayer}

	\foreach \x in {0,...,1}
	{
		\draw[thick, black, fill=orange] ($(loc)+(\x, -1)$) circle [radius=.1] node[] (C\x) {};
	}

	\foreach \x in {2,...,5}
	{
		\draw[thick, black, fill=magenta] ($(loc)+(\x, -1)$) circle [radius=.1] node (CL\x) {};
		\draw[thick, black, fill=green!60!black] ($(loc)+(\x, 0)$) circle [radius=.1] node (CR\x) {};
	}

	\draw[thick, black]
		(C0) to (C1)
		(C1) to[in=200, out=30] (CR2) node [above=.5cm, right=.1] {\tiny\texttt{testing}}
		(CR2) to (CR3) to (CR4) to (CR5)
		(C1) to (CL2) node[above=.4cm, right=.1] {\tiny\texttt{master}}
		(CL2) to (CL3) to (CL4) to (CL5)
	;

	\pause
	\draw[thick, black]
		(CR4) to[in=160, out=-20] (CL5)
	;
\end{tikzpicture}
}
\end{center}
\end{frame}

\subsection{Fork}
\begin{frame}[c, plain]{Begriff: Fork}
\begin{center}
\resizebox{!}{\textheight}{
\begin{tikzpicture}
	\visible<1->{
	% rick's repos
	\draw (0, 0)          pic{db} node (DBA1){} node[below=.2cm] {\texttt{proj}};
	\draw (DBA1) ++(6, 0) pic{db} node (DBA2){} node[below=.2cm] {\texttt{rick/proj}};
	}

	% morty's repos
	\visible<5->{
	\draw (DBA1) ++(0,-5) pic{db} node (DBB1){} node[below=.2cm] {\texttt{proj}};
	}
	\visible<4->{
	\draw (DBB1) ++(6, 0) pic{db} node (DBB2){} node[below=.2cm] {\texttt{morty/proj}};
	}
	

	\begin{pgfonlayer}{bg}
	\visible<1->{

	% laptop background
	\draw[blue!80, fill=blue!10!white]
		(DBA1.north west) ++(-1.2, 3) node[black, anchor=north west, below right=.2cm]{Laptop}
		rectangle ($(DBB1.south east) + (1.2, -1.5)$);

	% server background
	\draw[red!80, fill=red!10!white]
		(DBA2.north west) ++(-1.2, 3) node[black, anchor=north west, below right=.2cm]{Server}
		rectangle ($(DBB2.south east) + (1.2, -1.5)$);

	}

	% rick's background
	\visible<2->{
	\draw[magenta!80, fill=magenta!20, fill opacity=.8]
		(DBA1.north west) ++(-2.5, 2)
		node[black, anchor=north east, rotate=90, below left=.2cm] {Rick}
		rectangle ($(DBA2.south east) + (2, -.8)$);
	}

	% morty's background
	\visible<3->{
	\draw[cyan!80, fill=cyan!20, fill opacity=.8]
		(DBB1.north west) ++(-2.5, 2)
		node[black, anchor=north east, rotate=90, below left=.2cm] {Morty}
		rectangle ($(DBB2.south east) + (2, -.8)$);
	}
	\end{pgfonlayer}

	\visible<1->{
	% rick's arrow
	\draw[darkgray, ultra thick, <->]
		($(DBA1.north east)+(.75,.5)$) to node[pos=.5, above]
		{\footnotesize\texttt{push / pull}} ($(DBA2.north west)+(-.75,.5)$);
	}

	\visible<5->{
	% morty's arrow
	\draw[darkgray, ultra thick, <->]
		($(DBB1.north east)+(.75,.5)$) to node[pos=.5, above]
		{\footnotesize\texttt{push / pull}} ($(DBB2.north west)+(-.75,.5)$);
	}

	\visible<4->{
	% forking arrow
	\draw[black, ultra thick, ->]
		($(DBA2.south west)+(0,-.75)$) to node[
			pos=.5, left=.2cm, thick, draw=red, fill=white
		]{\textbf{\texttt{fork}}} ($(DBB2.north west)+(0,1.75)$);
	}

	% PR arrow
	\visible<6->{
	\draw[black, ultra thick, <-]
		($(DBA2.south east)+(0,-.75)$) to node[
			pos=.5, right=.2cm, thick, draw=red, fill=white
		]{\textbf{\texttt{pull request}}} ($(DBB2.north east)+(0,1.75)$);
	}

	% \visible<6->{
	% \draw[darkgray, ultra thick, ->]
		% ($(DBA2.south west)+(-1,-.5)$) to[bend right] node[pos=.6, below right, rotate=30]
		%{\texttt{pull}}($(DBB1.north east)+(.75,1.5)$);
	% }
\end{tikzpicture}}
\end{center}
\end{frame}

\section{Git benutzen}
\begin{frame}{Tips}
    % TODO
\end{frame}

\end{document}
